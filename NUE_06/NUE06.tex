% \newcommand{\prototitle}{Versuch 2 - Statistik}
% \newcommand{\Fachbereich}{Praktikum Messtechnik}
% \input{../packages/tu_header}

\newcommand{\institut}{Institut f\"ur Telekommunikationssysteme}
\newcommand{\fachgebiet}{Nachrichten\"ubertragung}
\newcommand{\veranstaltung}{Praktikum Nachrichten\"ubertragung}
\newcommand{\pdfautor}{\"Ozg\"u Dogan (326 048), Boris Henckell (325 779)}
\newcommand{\autor}{\"Ozg\"u Dogan (326 048)\\ Boris Henckell (325 779)}
\newcommand{\gruppe}{Gruppe: D03}
%\newcommand{\betreuer}{Betreuer: Mahmoud Felk}


\newcommand{\pdftitle}{Nachrichten\"ubertragung\ Praktikum\ 06}
\newcommand{\prototitle}{Praktikum 06 \\ Digitale Übertragungstechnik: Digitale Empfänger}


\input{../../packages/tu_header_8}
% \begin{document}

% \lstlistoflistings
\definecolor{darkgray}{rgb}{0.95,0.95,0.95}
\definecolor{darkolivegreen}{HTML}{01a801}
\definecolor{functionsBlue}{HTML}{32b9b9}
\definecolor{variableRed}{rgb}{1,0,0}
\definecolor{stringBrown}{HTML}{bc8e8e} % f geht nicht

\lstset{
        %\lstset{extendedchars=true} % Umlaute an der richtigen stelle und nicht am Anfang ausgeben
        %basicstyle=\footnotesize\ttfamily,
        basicstyle=\small,
        %
        inputencoding=utf8,
        %
        tabsize=4,
        showspaces=false,
        showtabs=false,
        showstringspaces=true, % no special string spaces
        %
        backgroundcolor=\color{darkgray}, % background
        stringstyle=\color{stringBrown}\fseries, % Strings
        keywordstyle=\color{functionsBlue}\bfseries, % keywords Blau
        identifierstyle=\color{variableRed}, % variablen
        commentstyle=\color{darkolivegreen}, %  comments
        %
        breaklines=true,
        %
        numbers=left,
        numberstyle=\tiny,
        stepnumber=1,
        numbersep=7pt,
        %
        frame=single,
        columns=flexible,
        %
        xleftmargin=-2cm,
        xrightmargin=-1.5cm,
        %
        language=Matlab
}

%---------------------------------------------------------------------
%---------------------------------------------------------------------
%---------------------------------------------------------------------


\section{Einleitung}
\begin{quote}
    
    \TODO{Einleitung schreiben} \\


\end{quote}%beende Einleitung
%--------------------------------------------------------------------
%--------------------------------------------------------------------    

\section{Motivation}
\begin{quote}
    
    
    
\end{quote} %section

%--------------------------------------------------------------------
%--------------------------------------------------------------------    


\section{Theorie}
\begin{quote}

    
    
    \end{quote}%section

%--------------------------------------------------------------------
%--------------------------------------------------------------------    
\section{Vorbereitungsaufgabe}
\begin{quote}
    
    Zunächst sollten wir uns als Vorbereitung für den Praktikumstermin mit dem
    Inhalt des Kapitels zu optimalen Empfängerstrukturen aus dem Skript vertraut
    machen und nachvollziehen können.\\
    
    Als nächstes sollte mit Hilfe einer Matlab Datei bipolare Sendeimpulse
    für eine bipolare Übertragung definiert werden. Diese Sendeimpulse sollten als
    Vektoren SF0 und SF1 mit $-1$ und $1$ ein Kreuzkorrelationsergebnis von $\rho
    = 0, -\frac{1}{3}$ oder $-1$ liefern, wobei die Vektoren für $\rho = -1$ und
    $\rho = -\frac{1}{3}$ drei Werte und für $\rho = 0$ vier Werte besitzen
    müssen.\\
    Als Kontrolle dafür, sollte in der Matlab-Datei tatsächlich die
    Kreuzkorrelation durchgeführt und das Ergebnis untersucht werden. Außerdem wird
    die Bitenergie $E_B$ der Sendeimpulse berechnet, wobei die Vektoren als
    zeitkontinuierliche Spannungsverläufe der Dauer $T_B$ angenommen wird, bei
    denen Die Spannungsamplituden nur $+1\ V$ oder $-1\ V$ betragen können.
    Es sollte beachtet werden, dass $T_{SF} = 20\mu s$ konstant bleiben und sich
    somit das $T_B$ mit $T_B = 2 \cdot N \cdot \T_{SF}$ mit N Werten in SF0 bzw.
    SF1 berechnen lässt. Als Bedingung sollte noch gelten, dass die Bitenergien für
    SF0 und SF1 als Paar für ein $\rho$ gleich sein sollten.\\
    
    Als Vorbereitung für den praktischen Teil des Versuchs, sollte eine Funktion
    SAF implementiert werden, welche einen SAF-Empfänger in Matlab realiseren soll.
    Der Input besteht aus DataSamples, ClkSamples und SFSamples, welche gegeben
    sind. der Output dagegen soll dabei der Vektor Values sein, in der die Werte
    nach dem signalangepasstem Filter eingetragen sein sollen.\\
    \TODO{Boris: kannst du hier noch genauer erklären wie die Funktion funzt?}
    
    
    Nach der Implementierung der SAF-Funktion sollten wir uns mit dem
    Versuchsaufbau und der Durchführung vertraut machen und  ein Blockschaltbild
    dazu entwerfen. Die genaue Vorgehensweise ist in dem Abschnitt Durchführung
    detaliert erläutert. Hier ist das entworfene Blockschaltbild zu dem Versuch:\\
    
    \begin{figure}[H]
        \centering
            \includegraphics[scale=0.45, trim = 1.5cm 15cm 1cm 0cm,
            clip]{./Bilder/Blockschaltbild.png}
                \caption{Blockschaltbild des Versuchsaufbau}
        \end{figure}
        
    
    Um die implementierte Funktion SAF zu überprüfen, führten wir mithilfe der
    vorgegebenen ENue\_SAF\_Messumgebung.m durch, welche das Hauptprogramm der
    Messung darstellt. Mit dem Parameter Simulation (für die Vorbereitung
    bleibt dieser bei $1$) kann man entscheiden, ob die Ausgabe und
    Einstellungen des Steckbretts simuliert werden, oder ob die Daten über die 
    D/A-Box ausgegeben und per PicoScope eingelesen werden. Mit dem Parameter
    SAF kann man einstellen, ob mit dem DataSignal eine Sendeformung und dem
    Empfangssignal eine signalangepasste Filterung durchgeführt wird. Weiterhin
    sollten wir für die Vorbereitung die Kanal- und Filtereinstellungen nicht
    verändern, wohingegen der NoiseFaktor, welcher den Faktor des Rauschens
    einstellt, variiert werden durfte um die Leistung AWGN (additive white
    Gaussain noise) des Kanals zu beeinflussen.\\
    Mit der Funktion Channel führt man die Übertragung entweder als
    Simulation oder auf dem ETT aus, wobei die Einstellungen der Simulation
    denen des wirklichen Aufbaus entsprechen. Es wird ein Rückgabevektor Y
    ausgegeben, welcher die Werte nach der Abtastung des Kanals enthält.
    Wenn SAF deaktiviert ist, wird eine einfache Nachabtastung mit einem
    Schwellwert von $0V$ durchgeführt.\\
    Außerdem entspricht der Rückgabewert Noise einer Rauschmessung des Kanals,
    wozu eine 0-Bit-Folge gesendet und das Empfangssignal ohne SAF oder
    Nachabtastung als PuciScope-Sample-Signal zurückgegeben wird. Dieses Signal
    enthält ungefähr $10000$ Samples.\\
    
    
    Zuletzt sollte beantwortet werden, wie der Multiplikationsfaktor für das
    Rauschen variiert werden müsste, damit die Wasserfallkurve ausreichend ermittelt werden kann.
    Im Versuch beträgt das Rauschen dabei $-6 dB$.
    
    \end{quote}%Theorie
    % beenden

%--------------------------------------------------------------------
%--------------------------------------------------------------------    

    
\section{Labordurchführung}
\begin{quote}
    
    \subsection{Aufgabe 2.1 - Aufbau des Versuches}
    \begin{quote}
        
        Die erste Aufgabe des Praktikums beschäftigt sich fast ausschließlich
        mit dem Aufbar des Versuches, um den Signalverlauf mit der Matlab
        Funktion\\ 
        ParallelOUT($[0 1 0 1 0 1 0],100$) zu untersuchen. Dazu wird
        die D/A-Box, welcher an den Computer angeschlossen und somit über
        die Matlab Dateien steuerbar ist, als Quelle für jegliche verwendete
        Signale verwendet. Die D/A-Box besitzt vier Ausgänge, die mit
        unterschiedlichen Farben gekennzeichnet sind. Der rote Ausgang gibt das
        DataSignal aus. Dieser kann eine Amplitude von $0V$ oder $5V$ besitzen.
        Um ihn auf den von uns erwünschte Spannungsbereich von $\pm 1V$ zu
        bringen, wird dieses Signal zunächst mit einer Gleichspannung von
        $-2.5V$ aus der variablen Spannungsquelle des Steckbretts addiert und
        danach mit einem Faktor von $\frac{2}{5}$ gedämpft. Damit erreichen wir
        den nötigen Spannungsbereich, welcher auf dem A Kanal des PicoScopes
        kontrolliert wird.\\
        Der blaue Ausgang der D/A-Box gibt das Clock-Signal wieder. Dieser wird
        auf den B Kanal des PicoScope geführt und ebenfalls kontrolliert.\\
        Als nächstes wird der PCM Decoder mit der D/A-Box verbunden, indem das
        Clock-Signal auch auf den Clock-Eingang des Decoder geführt wird.
        Weiterhin wird das Signal aus dem grünen Ausgang der Box, welcher das
        Frame-Signal (FS) wiedergibt, mit dem FS-Eingang des Decoders
        und die PCM-codierten Datenworte, die aus dem gelben Ausgang der Box
        entnommen werden können, mit dem PCM Data-Eingang des Decoders
        kontaktiert. Wichtig ist auch, dass der Schalter auf dem Decoder Modul
        auf PCM geschaltet ist und nicht auf TDM.\\
        Somit ist der PCM Decoder mit allen Signalen beliefert, die er zum
        decodieren braucht. Daher kann man nun das Output Signal verwenden,
        welcher eine Spannung des Faktors wiedergibt, die für die
        Verstärkung oder Dämpfung des Rauschens dient. Diese Spannung wird an
        einem Multiplikator mit dem $-6 dB$ Rauschen multipliziert und an dem
        Addierer mit Multiplikatoren mit dem DataSignal, welcher bereits auf den korrekten Spannungsbereich
        eingestellt wurde, addiert. Das Ergebniss dieser Addition wird weiterhin
        auf den A Kanal des PicosScopes geführt und ausgewertet.\\
        
        Zuletzt wird das verrauschte Signal überprüft, indem mit der Funktion
        PCM\_Decod(192) ein Verstärkungsfaktor für das Rauschen gesetzt wird.
        Das verrauschte Signal wird mithilfe der PicoScope-Software und der
        Matlab-Funktion ParallelOUT($[0 1 0 1 0 1 0],100$) dargestellt und auch
        mit den Werten $0$, $128$ und $255$ untersucht.
        
        \TODO{Foto vom Aufbau einfügen}
        
    \end{quote}%Ende Durchführung 1
    
    \subsection{Aufgabe 2.2 - Bitfehlermessung}
    \begin{quote}
    
        jfdk
    
    \end{quote}%Ende Durchführung 2

\end{quote}%beende Labordurchführung

%--------------------------------------------------------------------
%--------------------------------------------------------------------    

    
\section{Auswertung}
\begin{quote}
    
    \subsection{Vorbereitungsaufgabe}
    \begin{quote}
        
    \end{quote}  % Ende Subsection Vorbereitungsaufgabe
    
    \subsection{Aufgabe 2.1 - Aufbau des Versuches}
    \begin{quote}
        
    \end{quote}  % Ende Auswertung Aufbau des Versuches
    
    \subsection{Aufgabe 2.2 - Bitfehlermessung}
    \begin{quote}
        \TODO{Aufgabe 2.2} \\
    \end{quote}  % Ende Auswertung Bitfehlermessung
            
\end{quote}%beende Auswertung

%--------------------------------------------------------------------
%-------------------------------------------------------------------- 
    
\section{Zusammenfassung}
\begin{quote}

    \TODO{Zusammenfassug schreiben} \\
\end{quote}%beende Zusammenfassung
         

%--------------------------------------------------------------------
%--------------------------------------------------------------------    


\begin{thebibliography}{999}
%      \bibitem {PCM-Uebertragung} Prof. Dr.-Ing. Sikora, Thomas; Prof. Dr.-Ing. Noll, Peter: Einführung in die
%      Nachrichtenübertragung, S.272
%     \bibitem {Digitalisierung_des_Signals} Prof. Dr.-Ing. Sikora, Thomas; Prof. Dr.-Ing. Noll, Peter: Einführung in die
%      Nachrichtenübertragung, S.273
%     \bibitem {PCM_Decodierung} Prof. Dr.-Ing. Sikora, Thomas; Prof. Dr.-Ing. Noll, Peter: Einführung in die
%      Nachrichtenübertragung, S.276
%      



%Name, Vorname.; evtl. Name2, Vorname2.: Titel des Dokumentes
%oder Buches, Zeitschrift/Verlag/URL (Auflage, Erscheinungsort, -jahr), ggf. Seitenzahlen
% \bibitem {PasevalscheTheorem} \url{https://de.wikipedia.org/wiki/Parsevalsches_Theorem}, Zugriff
% 23.05.2012
\end{thebibliography}

\end{document}
        
