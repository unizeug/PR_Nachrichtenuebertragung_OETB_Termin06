% \newcommand{\prototitle}{Versuch 2 - Statistik}
% \newcommand{\Fachbereich}{Praktikum Messtechnik}
% \input{../packages/tu_header}

\newcommand{\institut}{Institut f\"ur Telekommunikationssysteme}
\newcommand{\fachgebiet}{Nachrichten\"ubertragung}
\newcommand{\veranstaltung}{Praktikum Nachrichten\"ubertragung}
\newcommand{\pdfautor}{\"Ozg\"u Dogan (326 048), Boris Henckell (325 779)}
\newcommand{\autor}{\"Ozg\"u Dogan (326 048)\\ Boris Henckell (325 779)}
\newcommand{\gruppe}{Gruppe: D03}
%\newcommand{\betreuer}{Betreuer: Mahmoud Felk}


\newcommand{\pdftitle}{Nachrichten\"ubertragung\ Praktikum\ 06}
\newcommand{\prototitle}{Praktikum 06 \\ Digitale Übertragungstechnik: Digitale Empfänger}


\input{../../packages/tu_header_8}
% \begin{document}

% \lstlistoflistings
\definecolor{darkgray}{rgb}{0.95,0.95,0.95}
\definecolor{darkolivegreen}{HTML}{01a801}
\definecolor{functionsBlue}{HTML}{32b9b9}
\definecolor{variableRed}{rgb}{1,0,0}
\definecolor{stringBrown}{HTML}{bc8e8e} % f geht nicht

\lstset{
        %\lstset{extendedchars=true} % Umlaute an der richtigen stelle und nicht am Anfang ausgeben
        %basicstyle=\footnotesize\ttfamily,
        basicstyle=\small,
        %
        inputencoding=utf8,
        %
        tabsize=4,
        showspaces=false,
        showtabs=false,
        showstringspaces=true, % no special string spaces
        %
        backgroundcolor=\color{darkgray}, % background
        stringstyle=\color{stringBrown}\fseries, % Strings
        keywordstyle=\color{functionsBlue}\bfseries, % keywords Blau
        identifierstyle=\color{variableRed}, % variablen
        commentstyle=\color{darkolivegreen}, %  comments
        %
        breaklines=true,
        %
        numbers=left,
        numberstyle=\tiny,
        stepnumber=1,
        numbersep=7pt,
        %
        frame=single,
        columns=flexible,
        %
        xleftmargin=-2cm,
        xrightmargin=-1.5cm,
        %
        language=Matlab
}

%---------------------------------------------------------------------
%---------------------------------------------------------------------
%---------------------------------------------------------------------


\section{Einleitung}
\begin{quote}
	
	\TODO{Einleitung schreiben} \\


\end{quote}%beende Einleitung
%--------------------------------------------------------------------
%--------------------------------------------------------------------    

\section{Motivation}
\begin{quote}
	
	
	
\end{quote} %section

%--------------------------------------------------------------------
%--------------------------------------------------------------------    


\section{Theorie}
\begin{quote}

	
	
	\end{quote}%section

%--------------------------------------------------------------------
%--------------------------------------------------------------------    
\section{Vorbereitungsaufgabe}
\begin{quote}
	
	
	
\end{quote}%Theorie beenden

%--------------------------------------------------------------------
%--------------------------------------------------------------------    

    
\section{Labordurchführung}
\begin{quote}

   

\end{quote}%beende Labordurchführung

%--------------------------------------------------------------------
%--------------------------------------------------------------------    

    
\section{Auswertung}
\begin{quote}
    
    \subsection{Vorbereitungsaufgabe}
    \begin{quote}
        
    \end{quote}  % Ende Subsection Vorbereitungsaufgabe
    
    \subsection{Aufgabe 2.1 - Aufbau des Versuches}
    \begin{quote}
        
        Die erste Aufgabe des Praktikums beschäftigt sich fast ausschließlich
        mit dem Aufbar des Versuches, um den Signalverlauf mit der Matlab
        Funktion ParallelOUT($[0 1 0 1 0 1 0],100$) zu untersuchen. Dazu wird
        die D/A-Box, welcher an den Computer angeschlossen und somit über
        die Matlab Dateien steuerbar ist, als Quelle für jegliche verwendete
        Signale verwendet. Die D/A-Box besitzt vier Ausgänge, die mit
        unterschiedlichen Farben gekennzeichnet sind. Der rote Ausgang gibt das
        DataSignal aus. Dieser kann eine Amplitude von $0V$ oder $5V$ besitzen.
        Um ihn auf den von uns erwünschte Spannungsbereich von $\pm 1V$ zu
        bringen, wird dieses Signal zunächst mit einer Gleichspannung von
        $-2.5V$ aus der variablen Spannungsquelle des Steckbretts addiert und
        danach mit einem Faktor von $\frac{2}{5}$ gedämpft. Damit erreichen wir
        den nötigen Spannungsbereich, welcher auf dem A Kanal des PicoScopes
        kontrolliert wird.\\
        Der blaue Ausgang der D/A-Box gibt das Clock-Signal wieder. Dieser wird
        auf den B Kanal des PicoScope geführt und ebenfalls kontrolliert.\\
        Als nächstes wird der PCM Decoder mit der D/A-Box verbunden, indem das
        Clock-Signal auch auf den Clock-Eingang des Decoder geführt wird.
        Weiterhin wird das Signal aus dem grünen Ausgang der Box, welcher das
        Frame-Signal (FS) wiedergibt, mit dem FS-Eingang des Decoders
        und die PCM-codierten Datenworte, die aus dem gelben Ausgang der Box
        entnommen werden können, mit dem PCM Data-Eingang des Decoders
        kontaktiert. Wichtig ist auch, dass der Schalter auf dem Decoder Modul
        auf PCM geschaltet ist und nicht auf TDM.\\
        Somit ist der PCM Decoder mit allen Signalen beliefert, die er zum
        decodieren braucht. Daher kann man nun das Output Signal verwenden,
        welcher eine Spannung des Faktors wiedergibt, die für die
        Verstärkung oder Dämpfung des Rauschens dient. Diese Spannung wird an
        einem Multiplikator mit dem $-6 dB$ Rauschen multipliziert und an dem
        Addierer mit Multiplikatoren mit dem DataSignal, welcher bereits auf den korrekten Spannungsbereich
        eingestellt wurde, addiert. Das Ergebniss dieser Addition wird weiterhin
        auf den A Kanal des PicosScopes geführt und ausgewertet.\\
        
        Zuletzt wird das verrauschte Signal überprüft, indem mit der Funktion
        PCM\_Decod(192) ein Verstärkungsfaktor für das Rauschen gesetzt wird.
        Das verrauschte Signal wird mithilfe der PicoScope-Software und der
        Matlab-Funktion ParallelOUT($[0 1 0 1 0 1 0],100$) dargestellt und auch
        mit den Werten $0$, $128$ und $255$ untersucht.
        
    \end{quote}  % Ende Subsection Aufbau des Versuches
    
    \subsection{Aufgabe 2.2 - Bitfehlermessung}
    \begin{quote}
        \TODO{Aufgabe 2.2} \\
    \end{quote}  % Ende Subsection Quantisierungsfehler
         	
\end{quote}%beende Auswertung

%--------------------------------------------------------------------
%-------------------------------------------------------------------- 
    
\section{Zusammenfassung}
\begin{quote}

    \TODO{Zusammenfassug schreiben} \\
\end{quote}%beende Zusammenfassung
         

%--------------------------------------------------------------------
%--------------------------------------------------------------------    


\begin{thebibliography}{999}
%      \bibitem {PCM-Uebertragung} Prof. Dr.-Ing. Sikora, Thomas; Prof. Dr.-Ing. Noll, Peter: Einführung in die
%      Nachrichtenübertragung, S.272
%     \bibitem {Digitalisierung_des_Signals} Prof. Dr.-Ing. Sikora, Thomas; Prof. Dr.-Ing. Noll, Peter: Einführung in die
%      Nachrichtenübertragung, S.273
%     \bibitem {PCM_Decodierung} Prof. Dr.-Ing. Sikora, Thomas; Prof. Dr.-Ing. Noll, Peter: Einführung in die
%      Nachrichtenübertragung, S.276
%      



%Name, Vorname.; evtl. Name2, Vorname2.: Titel des Dokumentes
%oder Buches, Zeitschrift/Verlag/URL (Auflage, Erscheinungsort, -jahr), ggf. Seitenzahlen
% \bibitem {PasevalscheTheorem} \url{https://de.wikipedia.org/wiki/Parsevalsches_Theorem}, Zugriff
% 23.05.2012
\end{thebibliography}

\end{document}
  	    
